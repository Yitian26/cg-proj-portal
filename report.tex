% !TEX program = xelatex
\documentclass[11pt,a4paper]{article}

% ================= 页面与字体 =================
\usepackage{geometry}
\geometry{margin=2.4cm}

% XeLaTeX 中文支持
\usepackage{fontspec}
\usepackage{xeCJK}
\setmainfont{Times New Roman}

% 下面字体请按你电脑实际情况选一个能用的:
% Windows 常见:SimSun / Microsoft YaHei / SimHei
% Mac 常见:Songti SC / PingFang SC / STSong
\setCJKmainfont{SimSun}

% 段落与排版
\usepackage{setspace}
\setstretch{1.25}
\usepackage{indentfirst}
\setlength{\parindent}{2em}

% ================= 常用宏包 =================
\usepackage{graphicx}
\usepackage{float}
\usepackage{amsmath, amssymb}
\usepackage{booktabs}
\usepackage{subcaption}
\usepackage{enumitem}
\usepackage{multirow}
\usepackage{array}
\usepackage{url}

% 超链接(尽量靠后)
\usepackage{hyperref}
\hypersetup{
  colorlinks=true,
  linkcolor=black,
  urlcolor=black,
  citecolor=black
}

% 页眉页脚(可选,显得“像报告”)
\usepackage{fancyhdr}
\pagestyle{fancy}
\fancyhf{}
\lhead{计算机图形学大作业技术报告}
\rhead{Portal 三维解谜游戏}
\cfoot{\thepage}

% ================= 代码高亮 =================
\usepackage{listings}
\usepackage{xcolor}

\definecolor{backcolour}{rgb}{0.97,0.97,0.96}
\definecolor{codegray}{rgb}{0.45,0.45,0.45}
\definecolor{codegreen}{rgb}{0,0.5,0}
\definecolor{codepurple}{rgb}{0.55,0,0.8}

% listings 对中文不友好:建议代码里尽量英文注释;正文里用中文解释
\lstset{
  language=C++,
  backgroundcolor=\color{backcolour},
  basicstyle=\ttfamily\footnotesize,
  numbers=left,
  numberstyle=\tiny\color{codegray},
  stepnumber=1,
  numbersep=6pt,
  frame=single,
  rulecolor=\color{codegray},
  breaklines=true,
  breakatwhitespace=false,
  tabsize=4,
  showstringspaces=false,
  commentstyle=\color{codegreen},
  keywordstyle=\color{blue},
  stringstyle=\color{codepurple},
  columns=fullflexible
}

% ================= 标题信息 =================
\title{\textbf{基于自研 OpenGL 引擎的 Portal 类三维解谜游戏系统}\\
\large 计算机图形学课程大作业技术报告}
\author{徐易天,曾翔}
\date{\today}

\begin{document}

\maketitle
\thispagestyle{empty}
\newpage

\tableofcontents
\newpage

% ============================================================
\section{引言与项目概述}

本项目目标是实现一个可游玩的 Portal 风格三维解谜关卡:玩家在实验室场景中移动探索,拾取方块并放置于按钮触发机关,利用两扇相互连接的传送门穿越空间并最终到达终点。\par
在图形学实现上,我们未使用 Unity/Unreal 等商业引擎,而是基于 OpenGL Core Profile + C++ 自行搭建渲染与交互框架,完成了模型加载、实时渲染、第一人称漫游、碰撞检测、触发器逻辑、传送门视窗渲染与传送等核心能力。\par
(本报告结构与表述参考了我们提交的初稿文档,并在此基础上补全实现细节与代码证据。)

\subsection{玩法流程与操作说明}

\begin{itemize}[leftmargin=2em]
  \item 移动:W/A/S/D;跳跃:Space;视角:鼠标移动;
  \item 拾取/放下方块:E(对准可拾取方块后按 E 进入抓取;再次按 E 放下);
  \item 发射传送门:鼠标左键/右键分别放置 Portal A / Portal B;
  \item 抓取状态下左键:对抓取物体施加向前推力并释放(用于抛掷/推动方块);
  \item 重置:T(将玩家位置重置到原点并清零速度,用于快速回到可测试状态)。
  \item 截图:P
\end{itemize}

\begin{table}[H]
\centering
\small
\begin{tabular}{@{}ll@{}}
\toprule
\textbf{交互键位} & \textbf{行为说明} \\
\midrule
W/A/S/D & 第一人称移动(带加速度/重力) \\
Space & 起跳(落地判定后允许) \\
E & 重力枪拾取/放下(弹簧阻尼跟随) \\ 
LMB/RMB & 放置 Portal A / Portal B;抓取状态下 LMB 会推开并释放方块 \\ 
T & 将玩家位置重置到原点并清零速度 \\
P & 截图,保存至/screenshots\\
\bottomrule
\end{tabular}
\end{table}

\subsection{对照课程要求的完成情况}

课程要求包含 OBJ 模型导入、光照材质、几何变换、漫游、动画/多帧绘制、以及可选扩展如实时碰撞与可玩游戏等。
本项目以“可玩 Portal 解谜游戏”作为综合实现载体,涵盖必做项并实现扩展项(实时碰撞、触发机关、传送门视窗渲染等)。

% ============================================================
\section{系统总体架构}

系统采用模块化、面向对象设计,将输入、逻辑、物理、渲染解耦,主循环按“输入 $\rightarrow$ 更新 $\rightarrow$ 物理/触发 $\rightarrow$ 渲染”的顺序驱动。

\subsection{核心模块划分}

\begin{enumerate}[leftmargin=2em]
  \item \textbf{Application}:窗口与 OpenGL 上下文创建,主循环(deltaTime 计算、调用 update/render)。
  \item \textbf{Scene / GameObject}:场景管理与对象容器,统一管理模型资源、对象实例、Portal、Trigger、光源等。
  \item \textbf{Renderer}:封装 Shader、FBO、Portal 多次渲染、材质绑定与绘制流程。
  \item \textbf{Model/Mesh/Texture}:OBJ/MTL 解析、顶点数据组织、纹理加载与缓存。
  \item \textbf{PhysicsSystem}:刚体、重力、碰撞检测(SAT OBB)、raycast、触发器检测。
  \item \textbf{Player / PortalGun / Button / Flip / Trigger}:玩家控制、抓取系统、按钮触发器与翻转机关(Flip)、终点判定等玩法对象。
\end{enumerate}

\subsection{运行主循环结构}

主循环体现实时交互程序的经典结构:计算帧间隔 dt,处理输入、更新逻辑、渲染一帧并交换缓冲。

\begin{lstlisting}[caption={src/Application.cpp:主循环骨架(节选,和最终版本一致)}]
while (!glfwWindowShouldClose(window)) {
    float currentFrame = (float)glfwGetTime();
    deltaTime = currentFrame - lastFrame;
    lastFrame = currentFrame;

    // Clamp dt to avoid physics explosions
    if (deltaTime > 0.1f) deltaTime = 0.1f;

    // 1) input
    input.update();
    processInput(deltaTime);

    // 2) update
    Camera &activeCamera = getActiveCamera();
    scene->update(deltaTime, activeCamera);

    // 3) render
    renderer->render(*scene, activeCamera);

    glfwSwapBuffers(window);
    glfwPollEvents();
}
\end{lstlisting}


% ============================================================
\section{资源管线:Blender 建模、OBJ/MTL 解析与场景构建}

\subsection{建模与资源组织}

所有场景与道具模型由 Blender 自行建模导出 OBJ(含法线/UV),材质信息通过 MTL 关联,纹理以图片文件形式存储并在加载阶段建立缓存。

本项目资源目录结构如下(与工程一致):
\begin{itemize}[leftmargin=2em]
  \item \texttt{resources/obj/level/}:关卡墙体、按钮(button\_flip / button\_goal)、可移动方块(movable)等
  \item \texttt{resources/obj/portal\_gun/}:Portal Gun 模型
  \item \texttt{resources/obj/portal\_cube/}:可拾取方块模型
  \item \texttt{resources/texture/}:贴图
  \item \texttt{resources/shader/}:顶点/片元着色器
\end{itemize}

\subsection{OBJ 解析:顶点/法线/UV 与面索引三角化}

我们没有使用第三方 OBJ 库,而在 \texttt{Model} 中实现了基础解析:读取 v/vn/vt 与 f。对 f(多边形面)采用扇形三角化(Fan Triangulation)转为三角形列表,并建立“唯一顶点”索引缓存以复用数据。

\begin{lstlisting}[caption={src/Model.cpp:读取 f 并进行三角化(节选)}]
else if (prefix == "f") {
    // parse all vertices in this face
    std::vector<unsigned int> faceIndices;
    std::string vStr;
    while (ss >> vStr) {
        // parse "v/vt/vn"
        unsigned int vIdx=0, vtIdx=0, vnIdx=0;
        // ... split by '/'
        // push unique vertex index
        faceIndices.push_back(getOrCreateUniqueIndex(vIdx, vtIdx, vnIdx));
    }

    // fan triangulation: (0,i,i+1)
    for (size_t i = 1; i + 1 < faceIndices.size(); ++i) {
        indices.push_back(faceIndices[0]);
        indices.push_back(faceIndices[i]);
        indices.push_back(faceIndices[i+1]);
    }
}
\end{lstlisting}

\subsection{模型包围盒与 OBB 构建}

为了服务碰撞检测与触发器,模型加载后计算 AABB(min/max bound),并在对象层根据缩放/旋转构建 OBB(有向包围盒)。这样可以更准确地区分不同形状物体的碰撞范围。

\begin{figure}[H]
    \centering
    \includegraphics[width=0.8\textwidth]{figures/obb_debug.png}
    
    \caption{图中线框为场景中的触发器(Trigger)感应范围。Trigger 系统底层复用了物理引擎的 OBB(有向包围盒) 数据结构与求交算法,此处通过可视化 Debug 视图验证了 OBB 空间计算的正确性。}
    \label{fig:obb_debug}
\end{figure}

\subsection{物理穿梭与碰撞剔除 (Collision Culling)}

除了视觉上的连通,Portal 机制还要求在物理上允许玩家“穿过”原本存在的墙壁。若不处理碰撞,玩家会被传送门背后的墙体阻挡。为此,我们结合 Trigger 系统 与 位掩码 (Bitmask) 技术实现了动态碰撞剔除。

\subsubsection{基于掩码的碰撞过滤}
我们在 \texttt{PhysicsSystem} 中引入了碰撞掩码机制。仅当两个物体的掩码进行按位与运算结果非零时($(Mask_A \ \& \ Mask_B) \neq 0$),才进行碰撞解算。

\begin{lstlisting}[caption={src/PhysicsSystem.cpp:基于掩码的碰撞过滤}]
// 物理引擎中的碰撞预判
if ((a.rigidBody->collisionMask & b.rigidBody->collisionMask) == 0) 
    continue; // 掩码不匹配,忽略碰撞(允许穿透)
\end{lstlisting}

\subsubsection{动态掩码切换逻辑}
当传送门生成时,会将附着的墙壁标记为 \texttt{PORTAL\_ON}。同时,我们在传送门前方设置了一个 \texttt{NearTrigger}。当玩家或物体进入该区域时,触发器回调函数会将其碰撞掩码修改为 \texttt{NEAR\_PORTAL},使其与墙壁的掩码运算结果为 0,从而实现“穿墙”。

\begin{lstlisting}[caption={src/Portal.cpp:利用 Trigger 动态修改碰撞掩码}]
// Portal 初始化时设置近距离触发器
nearTrigger->onEnter = [](GameObject *obj) {
    // 进入传送门附近,修改掩码,允许穿过宿主墙壁
    obj->setCollisionMask(COLLISION_MASK_NEARPORTAL);
};

nearTrigger->onExit = [](GameObject *obj) {
    // 离开传送门区域,恢复默认碰撞掩码
    obj->setCollisionMask(COLLISION_MASK_DEFAULT);
};
\end{lstlisting}

通过这种机制,我们确保了玩家仅在传送门开启且靠近时才能穿过墙壁,而在其他情况下依然受物理碰撞限制,完美解决了“
% ============================================================
\section{渲染系统设计与实现细节}

\subsection{基础渲染管线}

渲染模块负责:
\begin{itemize}[leftmargin=2em]
  \item Shader 管理:统一编译/链接、设置 uniform(模型矩阵、视图矩阵、投影矩阵、光源参数)。
  \item 深度测试、面剔除、混合:保证室内场景遮挡关系正确;必要物体支持透明混合。
  \item 纹理绑定与材质:从 MTL/Texture 管线读取 diffuse/specular 等贴图并绑定。
\end{itemize}

\vspace{0.5cm} % 空一点距离
\textbf{渲染管线逻辑流程:}

为了实现传送门的“画中画”效果,我们的渲染管线并非简单的单次绘制,而是采用了多遍渲染 (Multi-pass Rendering) 架构。具体流程如下:

\begin{enumerate}
    \item \textbf{预处理阶段 (Portal Pass)}:
    \begin{itemize}
        \item 遍历场景中的传送门,判断是否需要递归渲染。
        \item 计算 \textit{虚拟摄像机} 矩阵:将主摄像机相对于“入口门”的位姿变换到“出口门”空间。
        \item \textbf{离屏渲染}:绑定 FBO (Frame Buffer Object),将虚拟视角的场景渲染到纹理。在此过程中启用 \textit{斜投影裁剪} 以防止物体遮挡。
    \end{itemize}
    
    \item \textbf{主渲染阶段 (Main Pass)}:
    \begin{itemize}
        \item 切换回默认帧缓冲 (Default Framebuffer)。
        \item 绘制主视角的实验室场景(墙壁、方块等)。
        \item 绘制传送门网格:此时将预处理阶段生成的 FBO 纹理作为材质绑定到传送门表面,从而实现空间连通的视觉效果。
    \end{itemize}
\end{enumerate}

\subsection{帧缓冲(FBO)封装}

Portal 视窗渲染依赖离屏渲染到纹理。我们封装了 \texttt{FrameBuffer}:创建 FBO、颜色纹理附件、深度模板 RBO,并提供 Bind/Unbind/Rescale 接口。

\begin{lstlisting}[caption={src/FrameBuffer.cpp:FBO 初始化(节选)}]
glGenFramebuffers(1, &fbo);
glBindFramebuffer(GL_FRAMEBUFFER, fbo);

// color texture
glGenTextures(1, &textureColorBuffer);
glBindTexture(GL_TEXTURE_2D, textureColorBuffer);
glTexImage2D(GL_TEXTURE_2D, 0, GL_RGB, width, height, 0,
             GL_RGB, GL_UNSIGNED_BYTE, NULL);
glFramebufferTexture2D(GL_FRAMEBUFFER, GL_COLOR_ATTACHMENT0,
                       GL_TEXTURE_2D, textureColorBuffer, 0);

// depth-stencil renderbuffer
glGenRenderbuffers(1, &rbo);
glBindRenderbuffer(GL_RENDERBUFFER, rbo);
glRenderbufferStorage(GL_RENDERBUFFER, GL_DEPTH24_STENCIL8, width, height);
glFramebufferRenderbuffer(GL_FRAMEBUFFER, GL_DEPTH_STENCIL_ATTACHMENT,
                          GL_RENDERBUFFER, rbo);
\end{lstlisting}

\subsection{Portal 视窗渲染:虚拟摄像机与视图矩阵变换}

Portal 的“看见另一扇门后的世界”,本质是把主相机通过入口门坐标系映射到出口门坐标系,得到一台虚拟相机,从该虚拟相机渲染场景到纹理,再把纹理贴到入口门表面。\par
我们在 \texttt{Portal::getTransformedView} 中实现了该变换(包含 180 度翻转以保证朝向一致)。

\begin{lstlisting}[caption={src/Portal.cpp:计算 Portal 变换后的视图矩阵(原代码节选)}]
glm::mat4 Portal::getTransformedView(glm::mat4 view) {
    glm::mat4 camTransform = glm::inverse(view);

    glm::mat4 rotation180 = glm::rotate(glm::mat4(1.0f),
        glm::radians(180.0f), glm::vec3(0, 1, 0));

    glm::mat4 destView = otherModel * rotation180
        * glm::inverse(myModel) * camTransform;

    return glm::inverse(destView);
}
\end{lstlisting}

\begin{figure}[H]
    \centering
    \includegraphics[width=0.8\textwidth]{figures/portal_view.png}
    \caption{透过Portal A观察到的画面,实际上是基于虚拟摄像机在出口门Portal B位置渲染的结果。矩阵变换保证了视线穿过门后的几何连续性。}
    \label{fig:portal_view}
\end{figure}

\subsection{斜投影裁剪(Oblique Projection Clipping)}

如果仅用普通投影矩阵渲染 Portal 纹理,会出现“门后墙体遮挡/穿帮”等问题。为此我们根据门平面构造裁剪平面,并对投影矩阵近裁剪面做斜投影修改,使得虚拟相机只渲染门平面另一侧的内容。\par
该部分在 \texttt{Renderer} 中实现,核心是构造 $q$ 与 $c$ 并替换投影矩阵的第三列。

\begin{lstlisting}[caption={src/Renderer.cpp:斜投影裁剪矩阵修改(原代码节选)}]
glm::vec4 viewPlane = srcPortal.getPlaneEquation(destView);

glm::vec4 q = glm::inverse(obliqueProjection) * glm::vec4(
    (viewPlane.x > 0 ? 1 : -1),
    (viewPlane.y > 0 ? 1 : -1),
    1.0f, 1.0f);

glm::vec4 c = viewPlane * (2.0f / glm::dot(viewPlane, q));

obliqueProjection[0][2] = c.x - obliqueProjection[0][3];
obliqueProjection[1][2] = c.y - obliqueProjection[1][3];
obliqueProjection[2][2] = c.z - obliqueProjection[2][3];
obliqueProjection[3][2] = c.w - obliqueProjection[3][3];
\end{lstlisting}

\subsection{递归渲染:Portal 中的 Portal}

Portal 互相可见时会产生“镜廊/画中画”效果。本项目支持递归渲染,但出于性能考虑设定最大深度,例如 \texttt{MAX\_PORTAL\_RECURSION=3},防止无限递归导致帧率崩溃。

\begin{figure}[H]
    \centering
    \includegraphics[width=0.8\textwidth]{figures/portal_recursion.png}
    
    \caption{展示了当两扇传送门互为可见时产生的“无限镜廊”视觉效果。渲染管线通过递归调用 \texttt{renderPortal} 函数,实现了深度为 3 的嵌套视图绘制 (MAX\_PORTAL\_RECURSION = 3),有效验证了离屏渲染逻辑的完备性。}
    \label{fig:portal_recursion}
\end{figure}

% ============================================================
\section{物理系统与碰撞检测(SAT + OBB)}

\subsection{刚体与力学更新}

游戏中玩家与方块具有重力、速度与受力(抓取时的弹簧力、投掷时的推力等)。物理更新阶段对速度/位移积分,并在碰撞发生时做位置修正与速度处理。

\subsection{OBB 表示与分离轴定理(SAT)}

为支持旋转物体(例如被抓取的方块可能发生旋转),仅使用 AABB 难以提供稳定精确的碰撞结果。因此我们实现 OBB + SAT:\par
对两个 OBB,测试 15 个潜在分离轴(A 的 3 个轴、B 的 3 个轴、以及 9 个叉乘轴)。若存在任意轴使投影区间不重叠,则不碰撞;否则碰撞并取最小穿透深度作为修正方向。

\begin{lstlisting}[caption={src/PhysicsSystem.cpp:SAT 单轴测试(原代码节选)}]
static bool TestAxis(const glm::vec3 &axis,
                     const OBB &a, const OBB &b,
                     float &minPen, glm::vec3 &bestAxis) {
    float len2 = glm::dot(axis, axis);
    if (len2 < 1e-6f) return true; // ignore near-zero axis
    glm::vec3 n = axis / std::sqrt(len2);

    float aMin, aMax, bMin, bMax;
    ProjectOBB(n, a, aMin, aMax);
    ProjectOBB(n, b, bMin, bMax);

    float overlap = std::min(aMax, bMax) - std::max(aMin, bMin);
    if (overlap < 0.0f) return false;

    if (overlap < minPen) {
        minPen = overlap;
        bestAxis = n;
    }
    return true;
}
\end{lstlisting}

\begin{lstlisting}[caption={src/PhysicsSystem.cpp:SAT 15 轴检测(原代码节选)}]
bool PhysicsSystem::checkCollisionSAT(const OBB &a, const OBB &b,
                                      glm::vec3 &correction) {
    float minPen = 1e9f;
    glm::vec3 bestAxis(0.0f);

    // 1) A's local axes
    for (int i = 0; i < 3; i++)
        if (!TestAxis(a.axes[i], a, b, minPen, bestAxis)) return false;

    // 2) B's local axes
    for (int i = 0; i < 3; i++)
        if (!TestAxis(b.axes[i], a, b, minPen, bestAxis)) return false;

    // 3) 9 cross product axes
    for (int i = 0; i < 3; i++)
        for (int j = 0; j < 3; j++) {
            glm::vec3 axis = glm::cross(a.axes[i], b.axes[j]);
            if (!TestAxis(axis, a, b, minPen, bestAxis)) return false;
        }

    // correction points from A to B
    glm::vec3 d = b.center - a.center;
    if (glm::dot(d, bestAxis) < 0) bestAxis = -bestAxis;

    correction = -bestAxis * minPen;
    return true;
}
\end{lstlisting}

\subsection{Raycast:用于抓取、开门、Portal 放置}

射线检测贯穿多个玩法:
\begin{itemize}[leftmargin=2em]
  \item E 拾取:从相机位置沿前向发射射线,命中可抓取物体后进入抓取状态;
  \item F 开门:射线命中 Door 则执行 toggle;
  \item 鼠标放置 Portal:射线命中可放置墙面,计算门的位姿并生成/更新门。
\end{itemize}

% ============================================================
\section{Portal 传送:位置/速度/相机姿态的一致性}

Portal “传送”不仅是把位置挪到另一端,还需要处理:
\begin{itemize}[leftmargin=2em]
  \item 位置:入口局部坐标 $\rightarrow$ 180° 翻转 $\rightarrow$ 出口世界坐标;
  \item 速度:同样作为方向向量进行变换,保证动量方向正确;
  \item 相机:玩家视角 Front/Up/Right 同步变换,并处理 Roll 恢复,避免“歪脖子永久倾斜”。
\end{itemize}

\begin{lstlisting}[caption={src/Portal.cpp:传送位置与速度变换(原代码节选)}]
glm::vec4 relPos = glm::inverse(srcRot)
    * glm::vec4(obj->position - this->position, 1.0f);

glm::vec4 newWorldPos4 = glm::vec4(linkedPortal->position, 1.0f)
    + dstRot * (rot180 * relPos);

// push forward to avoid immediate re-trigger
glm::vec3 dstForward = glm::normalize(glm::vec3(
    dstRot * glm::vec4(0,0,1,0)));
obj->position = glm::vec3(newWorldPos4) + dstForward * 0.15f;

if (obj->rigidBody) {
    glm::vec4 relVel = glm::inverse(srcRot)
        * glm::vec4(obj->rigidBody->velocity, 0.0f);
    glm::vec4 newVel4 = dstRot * (rot180 * relVel);
    obj->rigidBody->velocity = glm::vec3(newVel4);
}
\end{lstlisting}

\begin{lstlisting}[caption={src/Portal.cpp:玩家相机朝向变换与 Roll 恢复(原代码节选)}]
glm::vec4 relFront = glm::inverse(srcRot)
    * glm::vec4(player->camera.Front, 0.0f);
glm::vec4 newFront4 = dstRot * (rot180 * relFront);
glm::vec3 newFront = glm::normalize(glm::vec3(newFront4));
player->camera.Front = newFront;

// recompute yaw/pitch
float pitch = glm::degrees(glm::asin(glm::clamp(newFront.y, -1.0f, 1.0f)));
float yaw   = glm::degrees(std::atan2(newFront.z, newFront.x));
player->camera.Yaw = yaw;
player->camera.Pitch = pitch;

// roll recovery timer
player->initialRoll = roll;
player->rollRecoveryDuration = abs(roll) / 180.0f * 0.6f + 0.2f;
player->rollRecoveryTimer = player->rollRecoveryDuration;
\end{lstlisting}

\begin{figure}[H]
    \centering
    % 左图:扔进去前
    \begin{subfigure}[b]{0.45\textwidth}
        \centering
        % 截一张方块正要掉进地板传送门的图
         \includegraphics[width=\textwidth]{figures/momentum_entry.png} 
        \caption{入口:物体具有向下的垂直速度}
    \end{subfigure}
    \hfill
    % 右图:飞出来后
    \begin{subfigure}[b]{0.45\textwidth}
        \centering
        % 截一张方块从墙面传送门飞出、有抛物线轨迹的图
         \includegraphics[width=\textwidth]{figures/momentum_exit.png}

        \caption{出口:速度向量被正确变换为水平向前}
    \end{subfigure}
    
    \caption{动量守恒验证:通过“抛射 (Fling)”实验证明,当物体穿过传送门时,其速度向量经过矩阵变换后,大小保持不变,方向相对于出口平面正确重定向。}
    \label{fig:momentum}
\end{figure}

% ============================================================
\section{交互系统:抓取、按钮、Flip 机关与终点}

\subsection{重力枪抓取(弹簧阻尼跟随)}

拾取不是简单把方块“绑死在相机前”,而是用弹簧阻尼力让物体平滑跟随且保留惯性,提升手感与真实感。

\begin{lstlisting}[caption={src/Player.cpp:E 拾取/放下(原代码节选)}]
if (input.isKeyPressed(GLFW_KEY_E)) {
    if (isGrabbing) {
        isGrabbing = false;
    } else {
        auto result = scene->physicsSystem->raycast(camera.Position,
                        camera.Front, 5.0f);
        if (result.hit && result.object && result.object->isTeleportable) {
            isGrabbing = true;
            grabbedObject = result.object;
        }
    }
}
\end{lstlisting}

\begin{lstlisting}[caption={src/Player.cpp:弹簧阻尼施力(原代码节选)}]
if (isGrabbing && grabbedObject) {
    glm::vec3 targetPos = position + camera.Front * 2.0f
        + glm::vec3(0.0f, height * 0.5f, 0.0f);

    if (glm::length(targetPos - grabbedObject->position) > 3.0f) {
        isGrabbing = false;
    } else {
        glm::vec3 springForce = (targetPos - grabbedObject->position) * 30.0f;
        glm::vec3 dampingForce = -grabbedObject->rigidBody->velocity * 5.0f;
        grabbedObject->rigidBody->addForce(springForce + dampingForce);
    }
}
\end{lstlisting}

\subsection{按钮与 Flip:Trigger 驱动的机关联动}

按钮(Button)为自身生成 Trigger 区域用于检测“有人/物压住”。Scene 在每帧更新中读取 Button 的按下/抬起沿边(getIsPressed / getIsReleased),进而驱动 Flip 机关翻转或复位;同时考虑了 Flip 旋转期间 Portal 贴附失效的处理,避免穿模与视角错误。

\begin{lstlisting}[caption={src/Button.cpp:为按钮生成 Trigger(节选)}]
std::unique_ptr<Trigger> Button::createTrigger() {
    glm::vec3 min = model->minBound * scale;
    glm::vec3 max = model->maxBound * scale;

    glm::vec3 center = initialPosition + (min + max) * 0.5f;
    glm::vec3 size = max - min;

    glm::vec3 triggerMin = center - size * 0.5f + glm::vec3(-0.2f, 0.0f, -0.2f);
    glm::vec3 triggerMax = center + size * 0.5f + glm::vec3(0.2f, 0.5f, 0.2f);

    return std::make_unique<Trigger>(triggerMin, triggerMax);
}
\end{lstlisting}

\begin{lstlisting}[caption={src/Scene.h:按钮驱动 Flip 与 Portal 失效处理(节选)}]
auto button_flip = objects.find("button_flip");
if (button_flip != objects.end()) {
    Button *btn = dynamic_cast<Button *>(button_flip->second.get());
    if (btn && btn->getIsPressed()) {
        auto flipWall = objects.find("flip_wall");
        if (flipWall != objects.end()) {
            Flip *flip = dynamic_cast<Flip *>(flipWall->second.get());
            if (flip) flip->flip();
        }
    } else if (btn && btn->getIsReleased()) {
        auto flipWall = objects.find("flip_wall");
        if (flipWall != objects.end()) {
            Flip *flip = dynamic_cast<Flip *>(flipWall->second.get());
            if (flip) flip->reset();
        }
    }
}
\end{lstlisting}

\subsection{终点判定与通关重置}

终点同样通过按钮(Button + Trigger)实现:Application 在主循环中检测终点按钮(button\_goal)的按下沿,一旦被触发则调用 \texttt{glfwSetWindowTitle(window, "You Win!")} 给出胜利反馈。

\begin{figure}[H]
    \centering
    % 左图:到达终点(游戏内画面)
    \begin{subfigure}[b]{0.48\textwidth}
        \centering
        \includegraphics[width=\textwidth]{figures/goal_reach.png} 
        \caption{终点平台}
        \label{fig:goal_visual}
    \end{subfigure}
    \hfill
    % 右图:标题栏反馈(UI 逻辑)
    \begin{subfigure}[b]{0.48\textwidth}
        \centering
        % 你的标题栏截图,建议截成长条形
        \includegraphics[width=\textwidth]{figures/win_titlebar.png}
        \caption{窗口标题栏显示通关状态(You Win!)}
        \label{fig:win_title}
    \end{subfigure}
    
    \caption{当玩家进入终点区域触发 Trigger 后,后端逻辑不仅在控制台输出日志,还通过 \texttt{glfwSetWindowTitle} 实时更新窗口标题,向玩家提供直观的胜利反馈。}
    \label{fig:game_win}
\end{figure}

% ============================================================
\section{关键工程问题与调试经验}

\subsection{OBJ 资源丢失与路径问题}

开发中常见问题:运行时报 \texttt{Failed to open OBJ file: ...},根因通常为工作目录与资源相对路径不一致。解决方式:
\begin{itemize}[leftmargin=2em]
  \item 统一可执行程序运行目录(例如把工作目录设为项目根);
  \item 或改为从配置文件读取资源根路径;
  \item 或使用 CMake 在构建时复制 resources 到输出目录。
\end{itemize}

\subsection{Portal 递归与性能权衡}

递归层数越深,FBO 渲染次数越多,GPU 开销成倍增加。我们通过限制最大递归深度并在门不可见/距离过远时跳过 Portal 渲染来平衡画面与帧率。
\subsection{传送后立即反复触发(“抖门/死循环”)}

典型现象:物体穿门后立刻又被入口触发器检测到,来回传送造成死循环。我们采用“前向推开”策略:传送到出口后沿门法线方向推离一小段距离(见前文 \texttt{teleportForwardPush}),并结合 nearTrigger/碰撞 mask 做短暂保护窗口,避免反复触发。

% ============================================================
\section{实验结果与展示}

为了直观展示本项目的核心功能与渲染效果,我们在实际运行中截取了以下关键场景,涵盖了从基础场景渲染、Portal 核心机制到物理交互与游戏通关的完整流程。

\begin{figure}[H]
    \centering
    % 第一行:场景与视窗
    \begin{subfigure}[b]{0.48\textwidth}
        \centering
        \includegraphics[width=\textwidth]{figures/scene_overview.png}
        \caption{实验室场景全景渲染(Blender 建模导入)}
        \label{fig:overview}
    \end{subfigure}
    \hfill
    \begin{subfigure}[b]{0.48\textwidth}
        \centering
        \includegraphics[width=\textwidth]{figures/portal_view.png}
        \caption{Portal 视窗效果:A 门表面实时显示 B 门视角}
        \label{fig:portal_view}
    \end{subfigure}
    
    \vspace{0.5cm} % 行间距
    
    % 第二行:递归与抓取
    \begin{subfigure}[b]{0.48\textwidth}
        \centering
        \includegraphics[width=\textwidth]{figures/portal_recursion.png}
        \caption{递归渲染:当双门互视时产生的“画中画”效果}
        \label{fig:recursion}
    \end{subfigure}
    \hfill
    \begin{subfigure}[b]{0.48\textwidth}
        \centering
        \includegraphics[width=\textwidth]{figures/cube_grab.png}
        \caption{物理交互:使用重力枪吸附并携带方块移动}
        \label{fig:grab}
    \end{subfigure}
    
    \vspace{0.5cm} % 行间距

    % 第三行:机关与通关
    \begin{subfigure}[b]{0.48\textwidth}
        \centering
        \includegraphics[width=\textwidth]{figures/puzzle_solve.png}
        \caption{逻辑解谜:方块压下按钮,触发 Flip 机关翻转打开通道}
        \label{fig:puzzle}
    \end{subfigure}
    \hfill
    \begin{subfigure}[b]{0.48\textwidth}
        \centering
        \includegraphics[width=\textwidth]{figures/win_titlebar.png}
        \caption{通关判定:到达终点区域后触发胜利逻辑}
        \label{fig:win}
    \end{subfigure}
    
    \caption{Portal 游戏系统核心功能综合演示}
    \label{fig:showcase}
\end{figure}
% ============================================================
\section{小组分工与协作方式}


\begin{table}[H]
\centering
\small
\begin{tabular}{@{}p{3.2cm}p{10.5cm}@{}}
\toprule
\textbf{成员} & \textbf{主要贡献} \\
\midrule
徐易天 & 憋憋 \\
曾翔 &  憋憋\\

\bottomrule
\end{tabular}
\end{table}

% ============================================================
\section{总结与展望}

通过本次大作业,我们完整实现了一个可玩的 Portal 类三维解谜关卡,覆盖从资源管线、实时渲染到物理交互与高级 Portal 技术的完整链路,达成课程要求并具备扩展性。
未来改进方向:
\begin{itemize}[leftmargin=2em]
  \item 更完善的光照与阴影(Shadow Mapping / SSAO),提升真实感;
  \item 使用 BVH/空间划分优化碰撞与 raycast 性能;
  \item 更丰富的机关组合与关卡编辑能力;
  \item Portal 放置限制与更完善的剪裁/模板缓冲方案对比实现。
\end{itemize}

% ============================================================
\appendix
\newpage
\section{附录:源代码清单}

\textbf{说明:}本附录通过 \texttt{\textbackslash lstinputlisting} 直接嵌入工程源代码,此处只展示关键代码,具体代码于src文件夹中。

\subsection{Application / 主循环}
\lstinputlisting[caption={src/Application.cpp}]{src/Application.cpp}
\lstinputlisting[caption={src/Application.h}]{src/Application.h}
%\lstinputlisting[caption={src/main.cpp}]{src/main.cpp}

\subsection{渲染与帧缓冲}
\lstinputlisting[caption={src/Renderer.cpp}]{src/Renderer.cpp}
\lstinputlisting[caption={src/Renderer.h}]{src/Renderer.h}
%\lstinputlisting[caption={src/FrameBuffer.cpp}]{src/FrameBuffer.cpp}
%\lstinputlisting[caption={src/FrameBuffer.h}]{src/FrameBuffer.h}
%\lstinputlisting[caption={src/Shader.cpp}]{src/Shader.cpp}
%\lstinputlisting[caption={src/Shader.h}]{src/Shader.h}
%\lstinputlisting[caption={src/Skybox.cpp}]{src/Skybox.cpp}
%\lstinputlisting[caption={src/Skybox.h}]{src/Skybox.h}

\subsection{Portal 系统}
\lstinputlisting[caption={src/Portal.cpp}]{src/Portal.cpp}
%\lstinputlisting[caption={src/Portal.h}]{src/Portal.h}
%\lstinputlisting[caption={src/PortalGun.h}]{src/PortalGun.h}

\subsection{物理与触发器}
\lstinputlisting[caption={src/PhysicsSystem.cpp}]{src/PhysicsSystem.cpp}
%\lstinputlisting[caption={src/PhysicsSystem.h}]{src/PhysicsSystem.h}
\lstinputlisting[caption={src/Trigger.cpp}]{src/Trigger.cpp}
\lstinputlisting[caption={src/Trigger.h}]{src/Trigger.h}

\subsection{玩家与交互}
\lstinputlisting[caption={src/Player.cpp}]{src/Player.cpp}
%\lstinputlisting[caption={src/Player.h}]{src/Player.h}
% 旧版本的 Door / GameMechanics 已在最终工程中移除或重构
%\lstinputlisting[caption={src/door.cpp}]{src/door.cpp}
%\lstinputlisting[caption={src/GameMechanics.h}]{src/GameMechanics.h}

\lstinputlisting[caption={src/Scene.h}]{src/Scene.h}
\lstinputlisting[caption={src/Button.cpp}]{src/Button.cpp}
\lstinputlisting[caption={src/Flip.cpp}]{src/Flip.cpp}
%\lstinputlisting[caption={src/InputManager.cpp}]{src/InputManager.cpp}
%\lstinputlisting[caption={src/InputManager.h}]{src/InputManager.h}

\subsection{模型与网格}
\lstinputlisting[caption={src/Model.cpp}]{src/Model.cpp}
%\lstinputlisting[caption={src/Model.h}]{src/Model.h}
%\lstinputlisting[caption={src/Mesh.cpp}]{src/Mesh.cpp}
%\lstinputlisting[caption={src/Mesh.h}]{src/Mesh.h}
%\lstinputlisting[caption={src/Texture.cpp}]{src/Texture.cpp}
%\lstinputlisting[caption={src/Texture.h}]{src/Texture.h}

\subsection{相机与对象基类}
\lstinputlisting[caption={src/Camera.cpp}]{src/Camera.cpp}
%\lstinputlisting[caption={src/Camera.h}]{src/Camera.h}
\lstinputlisting[caption={src/GameObject.cpp}]{src/GameObject.cpp}
%\lstinputlisting[caption={src/GameObject.h}]{src/GameObject.h}
%\lstinputlisting[caption={src/Scene.h}]{src/Scene.h}
%\lstinputlisting[caption={src/HUD.cpp}]{src/HUD.cpp}
%\lstinputlisting[caption={src/HUD.h}]{src/HUD.h}

\end{document}
